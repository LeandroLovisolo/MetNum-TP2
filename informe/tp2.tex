\parskip = 10pt

\newcommand{\real}{\mathbb{R}}

{\bf Introducci\'on}

La Transformada Discreta del Coseno  (DCT, por sus siglas en ingl\'es) es una herramienta que nos permite representar cualquier se\~nal en el plano de las frecuencias. Dado que es utilizada por el est\'andar de compresi\'on de im\'agenes JPEG y formato de video MPEG, se encuentra implementada en m\'as lugares de lo que pensamos: en cada c\'amara digital o tel\'efono m\'ovil. 
La DCT no solo tiene aplicaciones al mundo de la compresi\'on (donde los valores transformados pueden ser codificados de forma eficiente), sino tambi\'en al procesamiento: el an\'alisis de qu\'e frecuencias est\'an presentes en las se\~nales es esencial en ciertos contextos de aplicaci\'on.

La idea intuitiva de esta transformada, en el plano continuo, consiste en representar una funci\'on $f: \mathbb{R} \rightarrow \mathbb{R}$ en la base de funciones $\mathcal{B}=\{1, \cos(x), \cos(2x),...\}$.
En el plano discreto, la DCT se corresponde a un cambio de base: cada una de las funciones de la base $\mathcal{B}$ se discretiza en ciertos puntos pasando a ser una base de vectores en $\mathbb{R}^n$, (donde $n$ es la dimensi\'on del vector o se\~nal a transformar).
Es decir, dado un vector o se\~nal $x\in\mathbb{R}^n$ existe una matriz $M\in\mathbb{R}^{n\times n}$ de cambio de base que define la transformada DCT, donde $y=Mx$ es el vector o se\~nal transformado al espacio de frecuencias por la DCT (ver ap\'endice \ref{sec:dct}). Esta operaci\'on es f\'acilmente extensible a se\~nales de dos dimensiones (ver ap\'endice \ref{sec:dct2d}).


{\bf Enunciado}

El objetivo del trabajo es eliminar ruido sobre una se\~nal ruidosa $x\in\real^{n}$. Para ello se realiza el siguiente proceso: 
\begin{enumerate}
\item  $y:=Mx$ [Transformar usando Ec.~(\ref{eq:dctint}) de Ap. \ref{sec:dct}]
\item $\tilde{y} := f(y)$ [Modificar]
\item Resolver $M \tilde{x} = \tilde{y}$ [Reconstruir]
\end{enumerate}

 
Una forma de medir la calidad visual de la se\~nal reconstruida $\tilde{x}$, es a trav\'es del PSNR ({\em Peak Signal-to-Noise Ratio}).
EL PSNR es una m\'etrica `perceptual' (acorde a lo que perciben los humanos) y nos da una forma de medir la calidad de una imagen perturbada, siempre y cuando se cuente con la se\~nal original. 
Cuanto mayor es el PSNR, mayor es la calidad de la imagen. La unidad de medida es el decibel (db) y se considera que una diferencia de 0.5 db ya es notada por la vista humana. El PSNR se define como:
$$
\mathit{PSNR} = 10 \cdot \log_{10} \left( \frac{\mathit{MAX}^2_x}{\mathit{ECM}} \right)
$$
donde $\mathit{MAX}_x$ define el rango m\'aximo de la se\~nal (en caso de entradas de 8 bits sin signo, ser\'ia 255) y \emph{ECM} es el {\em error cuadr\'atico medio}, definido como:
$ \frac{1}{n} \sum_{i}{(x_{i} - \tilde{x}_{i})^2} $,
donde $n$ es la cantidad de elementos de la se\~nal, $x$ es la se\~nal original y $\tilde{x}$ es la se\~nal recuperada.

En la implementaci\'on realizada deben llevar a cabo los siguientes experimentos:
\begin{itemize}
\item Para varias se\~nales con distintos niveles de ruido, se deber\'an experimentar con al menos 2 estrategias (definiendo $f$ de forma conveniente) para modificar la se\~nal transformada $y$ (paso 2) con el objetivo de que la se\~nal recuperada $\tilde{x}$ contenga menos ruido; se deber\'an extraer conclusiones en cuanto a la calidad de la se\~nal recuperada, en funci\'on de la estrategia utilizada.
\item Se deber\'an repetir los anteriores experimentos tambi\'en sobre im\'agenes adaptando el m\'etodo para aplicar la transformada DCT en dos dimensiones seg\'un se explica en la ap\'endice \ref{sec:dct2d}. 
\item {\bf (Opcional)} Se deber\'a analizar la aplicaci\'on de la DCT `por bloques' sobre im\'agenes. Por ejemplo, si tenemos una imagen de $64\times 64$ p\'ixeles podemos subdividirla en: 4 bloques de $32\times 32$, o 16 bloques de $16\times 16$, o 64 bloques de $8\times 8$, y aplicar la DCT en 2D sobre cada uno de los bloques (considerar un tama\~no m\'inimo de $8\times 8$ para cada bloque).\\ 
Elegir una estrategia utilizada para se\~nales unidimiensionales y sacar conclusiones respondiendo a los siguientes interrogantes (realizando experimentos que justifiquen la respuesta): ?`Es lo mismo eliminar ruido sobre la imagen entera que de a bloques? ?`Qu\'e forma es m\'as conveniente en cuanto a la calidad visual? ?`Qu\'e forma es m\'as r\'apida? 
\end{itemize}

{\bf Formatos de archivos de entrada}

Las se\~nales ser\'an le\'idas de un archivo de texto en cuya primer l\'inea figuran la cantidad de datos y en la l\'inea siguiente se encuentran los datos en ASCII separados por espacios. Para leer y escribir im\'agenes sugerimos utilizar el formato {\em raw} binario \texttt{.pgm}\footnote{\url{http://netpbm.sourceforge.net/doc/pgm.html}}. 
El mismo es muy sencillo de implementar y compatible con muchos gestores de fotos\footnote{XnView \url{http://www.xnview.com/}} y Matlab.

\vskip 0.5 cm
\hrule
\vskip 0.1 cm

{\bf Fecha de entrega:} 
\begin{itemize}
\item \textsl{Formato electr\'onico:} jueves 16 de mayo de 2013, hasta las 23:59 hs., enviando el trabajo (informe+c\'odigo) a \texttt{metnum.lab@gmail.com}. El subject del email debe comenzar con el texto \verb|[TP2]| seguido de la lista de apellidos de los integrantes del grupo. 
\item \textsl{Formato f\'isico:} viernes 17 de abril de 2013, de 18 a 20hs (en la clase del labo).
\end{itemize}

\newpage
\appendix
\section{Transformada Coseno Discreta}\label{sec:dct}

Para generar la matriz $M\in\real^{n\times n}$ que define la transformada de Coseno Discreta definimos: 
\begin{itemize}
 \item Vector de frecuencias:  $g = \left(\begin{array}{c} 0 \\ 1 \\ \vdots \\ n-1 \end{array} \right)$
 \item Vector de muestreo: $ s= {\displaystyle \frac{\pi}{n} }\left(\begin{array}{c} \frac{1}{2} \\1+\frac{1}{2} \\  \vdots \\ (n-1)+\frac{1}{2} \end{array} \right) $
 \item Constante de normalizaci\'on: $C(k) = \left\{ \begin{array}{lr}\sqrt{\frac{1}{n}} & k=1 \\ \sqrt{\frac{2}{n}} & k > 1 \end{array} \right.$
\end{itemize}

Siendo $T=\cos(g\cdot s^t)$ la matriz resultante de aplicarle el coseno a cada elemento de la matriz $g\cdot s^t$,
finalmente definimos, $\widehat{M}_{i,j} = C(i) \cdot T_{i,j}$

Para obtener una versi\'on entera de la transformaci\'on que define la matriz $M$, la cual ser\'a aplicada a se\~nales (o vectores) enteras en el rango $[0,q]$, definimos:
\begin{equation}
M = \left\lfloor \frac{q \widehat{M} + 1}{2}  \right\rfloor \label{eq:dctint}
\end{equation} 
donde $\left\lfloor\cdot\right\rfloor$ indica la parte entera inferior\footnote{El redondeo de un n\'umero $m$ pude definirse como $\lfloor m + 1/2 \rfloor$. Luego, $M$ se define como el redondeo de $\widehat{M}\cdot(q/2)$.}. (Es decir, escalamos los elementos de la matriz $M$ por $q/2$ y luego redondeamos los valores.)

\subsection{Extensi\'on a 2D}\label{sec:dct2d}
Dada una matriz $B\in\real^{n\times n}$, podemos extender f\'acilmente la transformada DCT a se\~nales de dos dimensiones. Para ello, aplicamos la transformaci\'on por filas y por columnas: $ M\, B\, M^t $

