\documentclass[a4paper,10pt,twoside]{article}

\usepackage[top=1in, bottom=1in, left=1in, right=1in]{geometry}
\usepackage[utf8]{inputenc}
\usepackage[spanish,es-ucroman,es-noquoting]{babel}
\usepackage{setspace}
\usepackage{fancyhdr}
\usepackage{lastpage}
\usepackage{amsmath}
\usepackage{amsfonts}
\usepackage{verbatim}
\usepackage{float}
\usepackage{graphicx}
\usepackage{subcaption}
\usepackage{amssymb}
\usepackage{url}
\usepackage{moreverb}


% Evita que el documento se estire verticalmente para ocupar
% el espacio vacío en cada página.
\raggedbottom


%%%%%%%%%% Configuración de Fancyhdr - Inicio %%%%%%%%%%
\pagestyle{fancy}
\thispagestyle{fancy}
\lhead{Trabajo Práctico 2, Organización del Computador II}
\rhead{Capra, Lovisolo, Petaccio}
\renewcommand{\footrulewidth}{0.4pt}
\cfoot{\thepage /\pageref{LastPage}}

\fancypagestyle{caratula} {
   \fancyhf{}
   \cfoot{\thepage /\pageref{LastPage}}
   \renewcommand{\headrulewidth}{0pt}
   \renewcommand{\footrulewidth}{0pt}
}
%%%%%%%%%% Configuración de Fancyhdr - Fin %%%%%%%%%%


\begin{document}


%%%%%%%%%%%%%%%%%%%%%%%%%%%%%%%%%%%%%%%%%%%%%%%%%%%%%%%%%%%%%%%%%%%%%%%%%%%%%%%
%% Carátula                                                                  %%
%%%%%%%%%%%%%%%%%%%%%%%%%%%%%%%%%%%%%%%%%%%%%%%%%%%%%%%%%%%%%%%%%%%%%%%%%%%%%%%


\thispagestyle{caratula}

\begin{center}

\includegraphics[height=2cm]{DC.png} 
\hfill
\includegraphics[height=2cm]{UBA.jpg} 

\vspace{2cm}

Departamento de Computación,\\
Facultad de Ciencias Exactas y Naturales,\\
Universidad de Buenos Aires

\vspace{1cm}

\begin{spacing}{2.5}
\begin{Huge}
Reducción de Ruido con la \\Transformada Discreta del Coseno
\end{Huge}
\end{spacing}

\vspace{1cm}

Trabajo Práctico 1, \\
Métodos Numéricos, \\
Primer Cuatrimestre de 2013

\vspace{6cm}

\begin{tabular}{|c|c|c|}
\hline
Apellido y Nombre & LU & E-mail\\
\hline
María Candela Capra Coarasa & 234/11 & canduh\_27@hotmail.com\\
Leandro Lovisolo            & 645/11 & leandro@leandro.me\\
Lautaro José Petaccio       & 443/11 & lausuper@gmail.com\\
\hline
\end{tabular}

\end{center}
\vspace{1cm}

\textbf{Resumen:} \\
Se aplican distintos métodos para agregar ruido a señales sonoras e imágenes para luego, utilizando la Transformada Discreta del Coseno, plantear y sacar conclusiones sobre la eficiencia de varias técnicas de reducción del ruido agregado.

\textbf{Palabras clave:}
DCT, ruido, sonido, imágenes, PSNR, frecuencia.

\newpage


%%%%%%%%%%%%%%%%%%%%%%%%%%%%%%%%%%%%%%%%%%%%%%%%%%%%%%%%%%%%%%%%%%%%%%%%%%%%%%%
%% Índice                                                                    %%
%%%%%%%%%%%%%%%%%%%%%%%%%%%%%%%%%%%%%%%%%%%%%%%%%%%%%%%%%%%%%%%%%%%%%%%%%%%%%%%


\tableofcontents

\newpage


%%%%%%%%%%%%%%%%%%%%%%%%%%%%%%%%%%%%%%%%%%%%%%%%%%%%%%%%%%%%%%%%%%%%%%%%%%%%%%%
%% Introducción Teórica                                                      %%
%%%%%%%%%%%%%%%%%%%%%%%%%%%%%%%%%%%%%%%%%%%%%%%%%%%%%%%%%%%%%%%%%%%%%%%%%%%%%%%


\section{Introducción Teórica}

En este trabajo exploramos un conjunto de métodos basados en la Transformada Discreta del Coseno para eliminar ruidos aperiódicos sobre muestras de señales de una y dos dimensiones (ejemplo: audio e imágenes, respectivamente).

La Transformada Discreta del Coseno, de ahora en más DCT, permite expresar un vector en $\mathbb{R}^n$ como combinación lineal de vectores de la forma $\{(cos(0 * t_0), \ldots cos(0 * t_{n-1})), \ldots (cos(n-1 * t_0), \ldots cos(n-1 * t_{n-1})) \}$, donde n es el tamaño de la muestra y $t_i = (i + \frac{1}{2})\frac{\pi}{n}$. Gráficamente, se puede interpretar este cambio de base como la escritura de la muestra como una suma de cosenos de distintas frecuencias, donde las coordenadas del vector transformado son coeficientes que determinan la amplitud de cada coseno, y se presentan ordenados de menor a mayor frecuencia.

En el caso de señales bidimensionales representadas con matrices $\mathbb{R}^{n \times n}$, la matriz transformada obtenida es análoga al caso unidimensional, en la que se pueden obtener los coeficientes ordenados de menor a mayor frecuencia recorriendo las coordenadas diagonalmente de derecha a izquierda y arriba a abajo, partiendo de la coordenada superior izquierda.


%%%%%%%%%%%%%%%%%%%%%%%%%%%%%%%%%%%%%%%%%%%%%%%%%%%%%%%%%%%%%%%%%%%%%%%%%%%%%%%
%% Desarrollo                                                                %%
%%%%%%%%%%%%%%%%%%%%%%%%%%%%%%%%%%%%%%%%%%%%%%%%%%%%%%%%%%%%%%%%%%%%%%%%%%%%%%%


\section{Desarrollo}

Evaluamos dos métodos para reducir los tipos de ruido descritos a continuación. Para cada método realizamos una serie de experimentos con distintos tipos de muestras, y utilizamos la ecuación de \textit{peak signal-noise ratio}, de ahora en más PSNR\footnote{La fórmula del PSNR se encuentra en el apéndice A.} para medir el nivel de información recuperado. 

Los tipos de ruido tratados son los siguientes:

\begin{description}

\item[Ruido aditivo:]

Suma a cada elemento de la muestra una variable aleatoria. En nuestros experimentos, utilizamos variables aleatorias con distribución normal, con media cero y varianza 10.

Observamos que estos parámetros generan una deformación perceptible en todas nuestras muestras, pero no las distorsionan al punto de volverse irreconocibles.

\item[Ruido impulsivo:]

Reemplaza cada elemento de la muestra por $max$ o $min$ con probabilidad $p$, donde $max$ y $min$ representan el valor máximo y mínimo que adquieren los elementos de la muestra, y $p$ variable de acuerdo al experimento.

Utilizamos $p = 0.01$ en todos nuestros experimentos. Observamos que probabilidades más altas distorsionan demasiado las muestras y dificultan el análisis.

\end{description}

Presentamos los dos métodos evaluados a continuación.


%%%%%%%%%%%%%%%%%%%%%%%%%%%%%%%%%%%%%%%%%%%%%%%%%%%%%%%%%%%%%%%%%%%%%%%%%%%%%%%
%% Atenuar                                                                   %%
%%%%%%%%%%%%%%%%%%%%%%%%%%%%%%%%%%%%%%%%%%%%%%%%%%%%%%%%%%%%%%%%%%%%%%%%%%%%%%%


\subsection{Atenuar}

Dados una constante $k$ y un índice $i$, multiplicamos por $k$ los coeficientes de la señal transformada con índice mayor o igual a $i$.

Decidimos usar índices $i = n * 0.5$ e $i = n * 0.3$ para el caso de muestras unidimensionales y bidimensionales, respectivamente. Para las muestras evaluadas, estos índices maximizan el nivel de PSNR observados luego de aplicar el método. Al transformar estas muestras, observamos que el grueso de la información se encuentra en los coeficientes más bajos, y por lo tanto al atenuar los coeficientes correspondientes a frecuecias más elevadas se produce una mejora en la relación señal ruido dejando intactos los coeficientes con la mayoría de la información.

Similarmente, el $k$ elegido es $0.1$. Valores más altos no logran reducir significativamente el ruido, y anular los coeficientes de índice mayor a $i$ implica una importante pérdida de información.

Presentamos a continuación los resultados de distintos experimentos aplicando este método.

\subsubsection{Ruido aditivo en audio}

\begin{description}
  \item[Muestra:] ramp1234.txt
  \item[PSNR inicial:] 22.55 dB
  \item[PSNR final:] 25.03 dB
\end{description}

\begin{figure}[H]
  \centering
  \includegraphics[width=15cm]{graficos/ramp_aditivo_atenuar_muestra.png}
  \caption{Muestra en base canónica.}
\end{figure}

\begin{figure}[H]
  \centering
  \includegraphics[width=15cm]{graficos/ramp_aditivo_atenuar_dct.png} 
  \caption{Muestra en base DCT.}
\end{figure}


\begin{description}
  \item[Muestra:] dopp512.txt
  \item[PSNR inicial:] 14.02 dB
  \item[PSNR final:] 16.28 dB
\end{description}

\begin{figure}[H]
  \centering
  \includegraphics[width=15cm]{graficos/dopp_aditivo_atenuar_muestra.png}
  \caption{Muestra en base canónica.}
\end{figure}

\begin{figure}[H]
  \centering
  \includegraphics[width=15cm]{graficos/dopp_aditivo_atenuar_dct.png} 
  \caption{Muestra en base DCT.}
\end{figure}


\subsubsection{Ruido aditivo en imágenes}

\begin{description}
  \item[Muestra:] lena.pgm
  \item[PSNR inicial:] 27.23 dB
  \item[PSNR final:] 28.62 dB
\end{description}

\begin{figure}[H]
  \centering
  \begin{subfigure}[b]{0.45\textwidth}
    \centering
    \includegraphics[width=\textwidth]{graficos/lena_aditivo_muestra.png}    
    \caption{Antes de atenuar.}
  \end{subfigure}
  ~ 
  \begin{subfigure}[b]{0.45\textwidth}
    \centering
    \includegraphics[width=\textwidth]{graficos/lena_aditivo_atenuar_muestra.png}
    \caption{Después de atenuar.}
  \end{subfigure}
  \caption{Ruido aditivo en imágenes.}
\end{figure}

\begin{figure}[H]
  \centering
  \includegraphics[width=15cm]{graficos/lena_aditivo_atenuar_dct.png} 
  \caption{Muestra en base DCT.}
\end{figure}


\subsubsection{Ruido impulsivo en audio}

\begin{description}
  \item[Muestra:] dopp512.txt
  \item[PSNR inicial:] 15.54 dB
  \item[PSNR final:] 17.55 dB
\end{description}

\begin{figure}[H]
  \centering
  \includegraphics[width=15cm]{graficos/dopp_impulsivo_atenuar_muestra.png} 
  \caption{Muestra en base canónica.}
\end{figure}

\begin{figure}[H]
  \centering
  \includegraphics[width=15cm]{graficos/dopp_impulsivo_atenuar_dct.png} 
  \caption{Muestra en base DCT.}
\end{figure}


\subsubsection{Ruido impulsivo en imágenes}

\begin{description}
  \item[Muestra:] lena.pgm
  \item[PSNR inicial:] 21.53 dB
  \item[PSNR final:] 24.92 dB
\end{description}

\begin{figure}[H]
  \centering
  \begin{subfigure}[b]{0.45\textwidth}
    \centering
    \includegraphics[width=\textwidth]{graficos/lena_impulsivo_muestra.png}    
    \caption{Antes de atenuar.}
  \end{subfigure}
  ~ 
  \begin{subfigure}[b]{0.45\textwidth}
    \centering
    \includegraphics[width=\textwidth]{graficos/lena_impulsivo_atenuar_muestra.png}
    \caption{Después de atenuar.}
  \end{subfigure}
  \caption{Ruido impulsivo en imágenes.}
\end{figure}

\begin{figure}[H]
  \centering
  \includegraphics[width=15cm]{graficos/lena_impulsivo_atenuar_dct.png} 
  \caption{Muestra en base DCT.}
\end{figure}


%%%%%%%%%%%%%%%%%%%%%%%%%%%%%%%%%%%%%%%%%%%%%%%%%%%%%%%%%%%%%%%%%%%%%%%%%%%%%%%
%% Umbralizar                                                                %%
%%%%%%%%%%%%%%%%%%%%%%%%%%%%%%%%%%%%%%%%%%%%%%%%%%%%%%%%%%%%%%%%%%%%%%%%%%%%%%%


\subsection{Umbralizar}

Dado un umbral $\beta$ y un índice $i$, anulamos todos los coeficientes con índice mayor o igual a $i$ y valor absoluto menor a $\beta$.

Por los mismos motivos analizados en el método \textit{Atenuar}, usamos índices $i = n * 0.5$ e $i = n * 0.3$ para el caso de muestras unidimensionales y bidimensionales, respectivamente.

Despues de probar con distintos umbrales, logramos maximizar el PSNR obtenido utilizando $\beta = (max - min) * 0.1$ en la mayoría de los experimentos realizados, donde $max$ y $min$ representan el valor máximo y mínimo que adquieren los elementos de la muestra.


\subsubsection{Ruido aditivo en audio}

\begin{description}
  \item[Muestra:] ramp1234.txt
  \item[PSNR inicial:] 22.54 dB
  \item[PSNR final:] 24.60 dB
\end{description}

\begin{figure}[H]
  \centering
  \includegraphics[width=15cm]{graficos/ramp_aditivo_umbralizar_muestra.png}
  \caption{Muestra en base canónica.}
\end{figure}

\begin{figure}[H]
  \centering
  \includegraphics[width=15cm]{graficos/ramp_aditivo_umbralizar_dct.png} 
  \caption{Muestra en base DCT.}
\end{figure}


\begin{description}
  \item[Muestra:] dopp512.txt
  \item[PSNR inicial:] 14.31 dB
  \item[PSNR final:] 16.45 dB
\end{description}

\begin{figure}[H]
  \centering
  \includegraphics[width=15cm]{graficos/dopp_aditivo_umbralizar_muestra.png}
  \caption{Muestra en base canónica.}
\end{figure}

\begin{figure}[H]
  \centering
  \includegraphics[width=15cm]{graficos/dopp_aditivo_umbralizar_dct.png} 
  \caption{Muestra en base DCT.}
\end{figure}


\subsubsection{Ruido aditivo en imágenes}

\begin{description}
  \item[Muestra:] lena.pgm
  \item[PSNR inicial:] 27.27 dB
  \item[PSNR final:] 28.08 dB
\end{description}

\begin{figure}[H]
  \centering
  \begin{subfigure}[b]{0.45\textwidth}
    \centering
    \includegraphics[width=\textwidth]{graficos/lena_aditivo_muestra.png}    
    \caption{Antes de umbralizar.}
  \end{subfigure}
  ~ 
  \begin{subfigure}[b]{0.45\textwidth}
    \centering
    \includegraphics[width=\textwidth]{graficos/lena_aditivo_umbralizar_muestra.png}
    \caption{Después de umbralizar.}
  \end{subfigure}
  \caption{Ruido aditivo en imágenes.}
\end{figure}

\begin{figure}[H]
  \centering
  \includegraphics[width=15cm]{graficos/lena_aditivo_umbralizar_dct.png} 
  \caption{Muestra en base DCT.}
\end{figure}


\subsubsection{Ruido impulsivo en audio}

\begin{description}
  \item[Muestra:] dopp512.txt
  \item[PSNR inicial:] 14.87 dB
  \item[PSNR final:] 16.96 dB
\end{description}

\begin{figure}[H]
  \centering
  \includegraphics[width=15cm]{graficos/dopp_impulsivo_umbralizar_muestra.png}
  \caption{Muestra en base canónica.}
\end{figure}

\begin{figure}[H]
  \centering
  \includegraphics[width=15cm]{graficos/dopp_impulsivo_umbralizar_dct.png} 
  \caption{Muestra en base DCT.}
\end{figure}


\subsubsection{Ruido impulsivo en imágenes}

\begin{description}
  \item[Muestra:] lena.pgm
  \item[PSNR inicial:] 21.39 dB
  \item[PSNR final:] 24.92 dB
\end{description}

\begin{figure}[H]
  \centering
  \begin{subfigure}[b]{0.45\textwidth}
    \centering
    \includegraphics[width=\textwidth]{graficos/lena_impulsivo_muestra.png}    
    \caption{Antes de umbralizar.}
  \end{subfigure}
  ~ 
  \begin{subfigure}[b]{0.45\textwidth}
    \centering
    \includegraphics[width=\textwidth]{graficos/lena_impulsivo_umbralizar_muestra.png}
    \caption{Después de umbralizar.}
  \end{subfigure}
  \caption{Ruido impulsivo en imágenes.}
\end{figure}

\begin{figure}[H]
  \centering
  \includegraphics[width=15cm]{graficos/lena_impulsivo_umbralizar_dct.png} 
  \caption{Muestra en base DCT.}
\end{figure}


%%%%%%%%%%%%%%%%%%%%%%%%%%%%%%%%%%%%%%%%%%%%%%%%%%%%%%%%%%%%%%%%%%%%%%%%%%%%%%%
%% Conclusiones                                                              %%
%%%%%%%%%%%%%%%%%%%%%%%%%%%%%%%%%%%%%%%%%%%%%%%%%%%%%%%%%%%%%%%%%%%%%%%%%%%%%%%


\section{Conclusiones}

En el caso de las imágenes, observamos que al transformarlas al dominio de la DCT, los coeficientes de frecuencias más bajas contienen la mayoría de la información. Observamos además que los ruidos analizados modifican los coeficientes más altos de la imagen, donde se definen los detalles más finos de la misma. Esto nos permite reducir este tipo de ruido fácilmente sin una pérdida signficativa de información reduciendo o eliminando los coeficientes de frecuencias altas.

En contraste, los tipos de ruidos evaluados sobre señales unidimensionales se manifiestan en todos los coeficientes en el espacio de la DCT. No obstante, las muestras evaluadas presentan el grueso de la información en coeficientes medios y bajos, por lo que es posible alcanzar una mejora en PSNR atenuando los coeficientes de frecuencias más elevadas sin pérdidas significativas de información. Esto significa que la muestra resultante conserva el ruido original, pero con menor intensidad en las frecuencias más altas.

En cuanto a los métodos analizados, observamos incrementos de PSNR similares en ambos métodos, pero \textit{Atenuar} resulta superior a \textit{Umbralizar} en casi todos los experimentos, ya que conserva más información al atenuar coeficientes en lugar de eliminarlos completamente. 

En conclusión, los métodos basados en la DCT resultan una buena herramienta para eliminar estos tipos de ruido presentes en imágenes. Al aplicarlos en señales unidimensionales también se obtiene una mejora en el PSNR, pero menor que el caso bidimensional ya que no modifican el ruido en sus coeficientes más bajos.


%%%%%%%%%%%%%%%%%%%%%%%%%%%%%%%%%%%%%%%%%%%%%%%%%%%%%%%%%%%%%%%%%%%%%%%%%%%%%%%
%% Apéndice A: Enunciado del Trabajo Práctico                                %%
%%%%%%%%%%%%%%%%%%%%%%%%%%%%%%%%%%%%%%%%%%%%%%%%%%%%%%%%%%%%%%%%%%%%%%%%%%%%%%%

\newpage

\section{Apéndice A: Enunciado del Trabajo Práctico}

\input{enunciado}


%%%%%%%%%%%%%%%%%%%%%%%%%%%%%%%%%%%%%%%%%%%%%%%%%%%%%%%%%%%%%%%%%%%%%%%%%%%%%%%
%% Apéndice B: Código Fuente                                                 %%
%%%%%%%%%%%%%%%%%%%%%%%%%%%%%%%%%%%%%%%%%%%%%%%%%%%%%%%%%%%%%%%%%%%%%%%%%%%%%%%

\newpage

\section{Apéndice B: Código Fuente}

\subsection{Metodos.cpp}

\verbatimtabinput{../Metodos.cpp}

\subsection{Ecuaciones.cpp}

\verbatimtabinput{../Ecuaciones.cpp}

\subsection{Matriz.h}

\verbatimtabinput{../Matriz.h}

\subsection{Matriz.cpp}

\verbatimtabinput{../Matriz.cpp}


\end{document}