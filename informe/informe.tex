\documentclass[a4paper,10pt,twoside]{article}

\usepackage[top=1in, bottom=1in, left=1in, right=1in]{geometry}
\usepackage[utf8]{inputenc}
\usepackage[spanish,es-ucroman,es-noquoting]{babel}
\usepackage{setspace}
\usepackage{fancyhdr}
\usepackage{lastpage}
\usepackage{amsmath}
\usepackage{amsfonts}
\usepackage{verbatim}
\usepackage{graphicx}

%Configuraciones de tp2
\usepackage{amssymb}
\usepackage{url}

% Evita que el documento se estire verticalmente para ocupar
% el espacio vacío en cada página.
\raggedbottom


%%%%%%%%%% Configuración de Fancyhdr - Inicio %%%%%%%%%%
\pagestyle{fancy}
\thispagestyle{fancy}
\lhead{Trabajo Práctico 2, Organización del Computador II}
\rhead{Capra, Lovisolo, Petaccio}
\renewcommand{\footrulewidth}{0.4pt}
\cfoot{\thepage /\pageref{LastPage}}

\fancypagestyle{caratula} {
   \fancyhf{}
   \cfoot{\thepage /\pageref{LastPage}}
   \renewcommand{\headrulewidth}{0pt}
   \renewcommand{\footrulewidth}{0pt}
}
%%%%%%%%%% Configuración de Fancyhdr - Fin %%%%%%%%%%


\begin{document}
%%%%%%%%%%%%%%%%%%%%%%%%%%%%%%%%%%%%%%%%%%%%%%%%%%%%%%%%%%%%%%%%%%%%%%%%%%%%%%%
%% Carátula                                                                  %%
%%%%%%%%%%%%%%%%%%%%%%%%%%%%%%%%%%%%%%%%%%%%%%%%%%%%%%%%%%%%%%%%%%%%%%%%%%%%%%%


\thispagestyle{caratula}

\begin{center}

\includegraphics[height=2cm]{DC.png} 
\hfill
\includegraphics[height=2cm]{UBA.jpg} 

\vspace{2cm}

Departamento de Computación,\\
Facultad de Ciencias Exactas y Naturales,\\
Universidad de Buenos Aires

\vspace{1cm}

\begin{spacing}{2.5}
\begin{Huge}
Reducción de ruido con \\Transformada Discreta del Coseno
\end{Huge}
\end{spacing}

\vspace{1cm}

Trabajo Práctico 1, \\
Métodos Numéricos, \\
Primer Cuatrimestre de 2013

\vspace{6cm}

\begin{tabular}{|c|c|c|}
\hline
Apellido y Nombre & LU & E-mail\\
\hline
María Candela Capra Coarasa & 234/11 & canduh\_27@hotmail.com\\
Leandro Lovisolo            & 645/11 & leandro@leandro.me\\
Lautaro José Petaccio       & 443/11 & lausuper@gmail.com\\
\hline
\end{tabular}

\end{center}
\vspace{1cm}

\textbf{Resumen:} \\
Se aplican distintos métodos para agregar ruido a señales sonoras e imágenes para luego, utilizando la Transformada Discreta del Coseno, plantear y sacar conclusiones sobre la eficiencia de varias técnicas de reducción del ruido agregado.

\textbf{Palabras claves:}
Reducción, ruido, DST, sonido, imágenes, frecuencia.
\newpage

%%%%%%%%%%%%%%%%%%%%%%%%%%%%%%%%%%%%%%%%%%%%%%%%%%%%%%%%%%%%%%%%%%%%%%%%%%%%%%%
%% Índice                                                                    %%
%%%%%%%%%%%%%%%%%%%%%%%%%%%%%%%%%%%%%%%%%%%%%%%%%%%%%%%%%%%%%%%%%%%%%%%%%%%%%%%


\tableofcontents

\newpage

%%%%%%%%%%%%%%%%%%%%%%%%%%%%%%%%%%%%%%%%%%%%%%%%%%%%%%%%%%%%%%%%%%%%%%%%%%%%%%%
%% Introducción Teórica                                                      %%
%%%%%%%%%%%%%%%%%%%%%%%%%%%%%%%%%%%%%%%%%%%%%%%%%%%%%%%%%%%%%%%%%%%%%%%%%%%%%%%


\section{Introducción Teórica}

En este trabajo exploramos un conjunto de métodos basados en la Transformada Discreta del Coseno para eliminar ruidos aperiódicos sobre muestras de señales de una y dos dimensiones (ejemplo: audio e imágenes, respectivamente).

La Transformada Discreta del Coseno, de ahora en más DCT, permite expresar un vector en $\mathbb{R}^n$ como combinación lineal de vectores de la forma $\{(cos(0 * t_0), \ldots cos(0 * t_{n-1})), \ldots (cos(n-1 * t_0), \ldots cos(n-1 * t_{n-1})) \}$, donde n es el tamaño de la muestra y $t_i = (i + \frac{1}{2})\frac{\pi}{n}$. Gráficamente, se puede interpretar este cambio de base como la escritura de la muestra como una suma de cosenos de distintas frecuencias, donde las coordenadas del vector transformado son coeficientes que determinan la amplitud de cada coseno, y se presentan ordenados de menor a mayor frecuencia.

En el caso de señales bidimensionales representadas con matrices $\mathbb{R}^{n \times n}$, la matriz transformada obtenida es análoga al caso unidimensional, en la que se pueden obtener los coeficientes ordenados de menor a mayor frecuencia recorriendo las coordenadas diagonalmente de derecha a izquierda y arriba a abajo, partiendo de la coordenada superior izquierda.


% \textbf{Transformada Discreta del Coseno}
% En el caso real, expresa una función $f()$ determinada utilizando una base formada
% por $\{cos(0x),cos(x),cos(2x),...,cos((n-1)x)\}$.

% En el caso discreto es un cambio de base, dónde la base usada se compone de cosenos discretizados a distintas frecuencias


%%%%%%%%%%%%%%%%%%%%%%%%%%%%%%%%%%%%%%%%%%%%%%%%%%%%%%%%%%%%%%%%%%%%%%%%%%%%%%%
%% Desarrollo                                                                %%
%%%%%%%%%%%%%%%%%%%%%%%%%%%%%%%%%%%%%%%%%%%%%%%%%%%%%%%%%%%%%%%%%%%%%%%%%%%%%%%


\section{Desarrollo}
\subsection{Generación de ruidos}
Implementamos el ruido aditivo e impulsivo tanto para imágenes como para sonido.
\subsubsection{Ruido aditivo}
La implementación agrega ruido aditivo según la distribución probabilística normal pudiendo variarse su media y varianza para generar diferente cantidad de ruido.

Este ruido generalmente altera todas las frecuencias de la señal por igual variando su intensidad según los cambios en la varianza.

\subsubsection{Ruido impulsivo}
La implementación recorre la señal o la imágen y cambia sus valores por su máximo o mínimo según un valor $p$ asociado a los valores aleatorios obtenidos mediante una distribución probabilística uniforme entre $[0,1]$. Si el valor aleatorio es $< p$ entonces se impulsa la señal al máximo, si es $\geq p $ se impulsa al mínimo, y si cae entre ellos la señal queda intacta.

\subsection{Eliminación de ruido}
Implementamos y probamos los métodos de eliminación de ruido \"atenuar\" y \"umbralizar\" descriptos a continuación sobre intervalos, ya que, en las pruebas realizadas, se consiguieron mejoras en las señales en su aplicación a intervalos específicos.

En el caso de imágenes, el intervalo se tiene en cuenta según su recorrido en forma diagonal, abarcando desde los cosenos con más baja frecuencia hasta los de más alta.

\subsubsection{Atenuar intervalo}
Este método de reducción de ruido, multiplica a un intervalo de la señal por una constante k de doble precisión de punto flotante.

Para el caso de un k menor que uno, la señal se reduce, siendo útil para reducir incrementos de amplitudes en frecuencias.

\subsubsection{Umbralizar intervalo}
El método recorre un intervalo la señal, buscando y poniendo en 0 los valores que sean menores al valor absoluto de una constante k que recibe como parámetro.

La idea de su implementación es utilizarse para poder umbralizar ciertas frecuencias que no suelen pasar de un valor determinado, pudiendo atenuar el ruido de la señal.

%%%%%%%%%%%%%%%%%%%%%%%%%%%%%%%%%%%%%%%%%%%%%%%%%%%%%%%%%%%%%%%%%%%%%%%%%%%%%%%
%% Resultados                                                                %%
%%%%%%%%%%%%%%%%%%%%%%%%%%%%%%%%%%%%%%%%%%%%%%%%%%%%%%%%%%%%%%%%%%%%%%%%%%%%%%%


\section{Resultados}
Pendiente.

%%%%%%%%%%%%%%%%%%%%%%%%%%%%%%%%%%%%%%%%%%%%%%%%%%%%%%%%%%%%%%%%%%%%%%%%%%%%%%%
%% Discusión                                                                 %%
%%%%%%%%%%%%%%%%%%%%%%%%%%%%%%%%%%%%%%%%%%%%%%%%%%%%%%%%%%%%%%%%%%%%%%%%%%%%%%%


\section{Discusión}
Pendiente.

%%%%%%%%%%%%%%%%%%%%%%%%%%%%%%%%%%%%%%%%%%%%%%%%%%%%%%%%%%%%%%%%%%%%%%%%%%%%%%%
%% Conclusiones                                                              %%
%%%%%%%%%%%%%%%%%%%%%%%%%%%%%%%%%%%%%%%%%%%%%%%%%%%%%%%%%%%%%%%%%%%%%%%%%%%%%%%


\section{Conclusiones}
Pendiente.

%%%%%%%%%%%%%%%%%%%%%%%%%%%%%%%%%%%%%%%%%%%%%%%%%%%%%%%%%%%%%%%%%%%%%%%%%%%%%%%
%% Apéndice A: Enunciado del Trabajo Práctico                                %%
%%%%%%%%%%%%%%%%%%%%%%%%%%%%%%%%%%%%%%%%%%%%%%%%%%%%%%%%%%%%%%%%%%%%%%%%%%%%%%%

\newpage

\section{Apéndice A: Enunciado del Trabajo Práctico}
\input{tp2}


\end{document}