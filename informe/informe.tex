\documentclass[a4paper,10pt,twoside]{article}

\usepackage[top=1in, bottom=1in, left=1in, right=1in]{geometry}
\usepackage[utf8]{inputenc}
\usepackage[spanish,es-ucroman,es-noquoting]{babel}
\usepackage{setspace}
\usepackage{fancyhdr}
\usepackage{lastpage}
\usepackage{amsmath}
\usepackage{amsfonts}
\usepackage{verbatim}
\usepackage{float}
\usepackage{graphicx}
\usepackage{subcaption}
\usepackage{amssymb}
\usepackage{url}


% Evita que el documento se estire verticalmente para ocupar
% el espacio vacío en cada página.
\raggedbottom


%%%%%%%%%% Configuración de Fancyhdr - Inicio %%%%%%%%%%
\pagestyle{fancy}
\thispagestyle{fancy}
\lhead{Trabajo Práctico 2, Organización del Computador II}
\rhead{Capra, Lovisolo, Petaccio}
\renewcommand{\footrulewidth}{0.4pt}
\cfoot{\thepage /\pageref{LastPage}}

\fancypagestyle{caratula} {
   \fancyhf{}
   \cfoot{\thepage /\pageref{LastPage}}
   \renewcommand{\headrulewidth}{0pt}
   \renewcommand{\footrulewidth}{0pt}
}
%%%%%%%%%% Configuración de Fancyhdr - Fin %%%%%%%%%%


\begin{document}


%%%%%%%%%%%%%%%%%%%%%%%%%%%%%%%%%%%%%%%%%%%%%%%%%%%%%%%%%%%%%%%%%%%%%%%%%%%%%%%
%% Carátula                                                                  %%
%%%%%%%%%%%%%%%%%%%%%%%%%%%%%%%%%%%%%%%%%%%%%%%%%%%%%%%%%%%%%%%%%%%%%%%%%%%%%%%


\thispagestyle{caratula}

\begin{center}

\includegraphics[height=2cm]{DC.png} 
\hfill
\includegraphics[height=2cm]{UBA.jpg} 

\vspace{2cm}

Departamento de Computación,\\
Facultad de Ciencias Exactas y Naturales,\\
Universidad de Buenos Aires

\vspace{1cm}

\begin{spacing}{2.5}
\begin{Huge}
Reducción de Ruido con la \\Transformada Discreta del Coseno
\end{Huge}
\end{spacing}

\vspace{1cm}

Trabajo Práctico 1, \\
Métodos Numéricos, \\
Primer Cuatrimestre de 2013

\vspace{6cm}

\begin{tabular}{|c|c|c|}
\hline
Apellido y Nombre & LU & E-mail\\
\hline
María Candela Capra Coarasa & 234/11 & canduh\_27@hotmail.com\\
Leandro Lovisolo            & 645/11 & leandro@leandro.me\\
Lautaro José Petaccio       & 443/11 & lausuper@gmail.com\\
\hline
\end{tabular}

\end{center}
\vspace{1cm}

\textbf{Resumen:} \\
Se exploran dos métodos basados en la Transformada Discreta del Coseno para reducir distintos tipos de ruidos aperiódicos sobre señales de una y dos dimensiones.

\textbf{Palabras clave:}
DCT, ruido, sonido, imágenes, PSNR, frecuencia.
\newpage


%%%%%%%%%%%%%%%%%%%%%%%%%%%%%%%%%%%%%%%%%%%%%%%%%%%%%%%%%%%%%%%%%%%%%%%%%%%%%%%
%% Índice                                                                    %%
%%%%%%%%%%%%%%%%%%%%%%%%%%%%%%%%%%%%%%%%%%%%%%%%%%%%%%%%%%%%%%%%%%%%%%%%%%%%%%%


\tableofcontents

\newpage


%%%%%%%%%%%%%%%%%%%%%%%%%%%%%%%%%%%%%%%%%%%%%%%%%%%%%%%%%%%%%%%%%%%%%%%%%%%%%%%
%% Introducción Teórica                                                      %%
%%%%%%%%%%%%%%%%%%%%%%%%%%%%%%%%%%%%%%%%%%%%%%%%%%%%%%%%%%%%%%%%%%%%%%%%%%%%%%%


\section{Introducción Teórica}

En este trabajo exploramos un conjunto de métodos basados en la Transformada Discreta del Coseno para eliminar ruidos aperiódicos sobre muestras de señales de una y dos dimensiones (ejemplo: audio e imágenes, respectivamente).

La Transformada Discreta del Coseno, de ahora en más DCT, permite expresar un vector en $\mathbb{R}^n$ como combinación lineal de vectores de la forma $\{(cos(0 * t_0), \ldots cos(0 * t_{n-1})), \ldots (cos(n-1 * t_0), \ldots cos(n-1 * t_{n-1})) \}$, donde n es el tamaño de la muestra y $t_i = (i + \frac{1}{2})\frac{\pi}{n}$. Gráficamente, se puede interpretar este cambio de base como la escritura de la muestra como una suma de cosenos de distintas frecuencias, donde las coordenadas del vector transformado son coeficientes que determinan la amplitud de cada coseno, y se presentan ordenados de menor a mayor frecuencia.

En el caso de señales bidimensionales representadas con matrices $\mathbb{R}^{n \times n}$, la matriz transformada obtenida es análoga al caso unidimensional, en la que se pueden obtener los coeficientes ordenados de menor a mayor frecuencia recorriendo las coordenadas diagonalmente de derecha a izquierda y arriba a abajo, partiendo de la coordenada superior izquierda.


%%%%%%%%%%%%%%%%%%%%%%%%%%%%%%%%%%%%%%%%%%%%%%%%%%%%%%%%%%%%%%%%%%%%%%%%%%%%%%%
%% Desarrollo                                                                %%
%%%%%%%%%%%%%%%%%%%%%%%%%%%%%%%%%%%%%%%%%%%%%%%%%%%%%%%%%%%%%%%%%%%%%%%%%%%%%%%


\section{Desarrollo}

Evaluamos dos métodos para reducir los tipos de ruido descritos a continuación. Para cada método realizamos una serie de experimentos con distintos tipos de muestras, y utilizamos la ecuación de \textit{peak signal-noise ratio}, de ahora en más PSNR\footnote{La fórmula del PSNR se encuentra en el apéndice A.} para medir el nivel de información recuperado. 

Los tipos de ruido tratados son los siguientes:

\begin{description}

\item[Ruido aditivo:]

Suma a cada elemento de la muestra una variable aleatoria. En nuestros experimentos, utilizamos variables aleatorias con distribución normal, con media cero y varianza 10.

Observamos que estos parámetros generan una deformación perceptible en todas nuestras muestras, pero no las distorsionan al punto de volverse irreconocibles.

\item[Ruido impulsivo:]

Reemplaza cada elemento de la muestra por $max$ o $min$ con probabilidad $p$, donde $max$ y $min$ representan el valor máximo y mínimo que adquieren los elementos de la muestra, y $p$ variable de acuerdo al experimento.

Utilizamos $p = 0.01$ en todos nuestros experimentos. Observamos que probabilidades más altas distorsionan demasiado las muestras y dificultan el análisis.

\end{description}

Presentamos los dos métodos evaluados a continuación.


%%%%%%%%%%%%%%%%%%%%%%%%%%%%%%%%%%%%%%%%%%%%%%%%%%%%%%%%%%%%%%%%%%%%%%%%%%%%%%%
%% Atenuar                                                                   %%
%%%%%%%%%%%%%%%%%%%%%%%%%%%%%%%%%%%%%%%%%%%%%%%%%%%%%%%%%%%%%%%%%%%%%%%%%%%%%%%


\subsection{Atenuar}

Dados una constante $k$ y un índice $i$, multiplicamos por $k$ los coeficientes de la señal transformada con índice mayor o igual a $i$.

Decidimos usar índices $i = n * 0.5$ e $i = n * 0.3$ para el caso de muestras unidimensionales y bidimensionales, respectivamente. Para las muestras evaluadas, estos índices maximizan el nivel de PSNR observados luego de aplicar el método. Al transformar estas muestras, observamos que el grueso de la información se encuentra en los coeficientes más bajos, y por lo tanto al atenuar los coeficientes correspondientes a frecuecias más elevadas se produce una mejora en la relación señal ruido dejando intactos los coeficientes con la mayoría de la información.

Similarmente, el $k$ elegido es $0.1$. Valores más altos no logran reducir significativamente el ruido, y anular los coeficientes de índice mayor a $i$ implica una importante pérdida de información.

Presentamos a continuación los resultados de distintos experimentos aplicando este método.

\subsubsection{Ruido aditivo en audio}

\begin{description}
  \item[Muestra:] ramp1234.txt
  \item[PSNR inicial:] 22.55 dB
  \item[PSNR final:] 25.03 dB
\end{description}

\begin{figure}[H]
  \centering
  \includegraphics[width=15cm]{graficos/ramp_aditivo_atenuar_muestra.png}
  \caption{Muestra en base canónica.}
\end{figure}

\begin{figure}[H]
  \centering
  \includegraphics[width=15cm]{graficos/ramp_aditivo_atenuar_dct.png} 
  \caption{Muestra en base DCT.}
\end{figure}


\begin{description}
  \item[Muestra:] dopp512.txt
  \item[PSNR inicial:] 14.02 dB
  \item[PSNR final:] 16.28 dB
\end{description}

\begin{figure}[H]
  \centering
  \includegraphics[width=15cm]{graficos/dopp_aditivo_atenuar_muestra.png}
  \caption{Muestra en base canónica.}
\end{figure}

\begin{figure}[H]
  \centering
  \includegraphics[width=15cm]{graficos/dopp_aditivo_atenuar_dct.png} 
  \caption{Muestra en base DCT.}
\end{figure}


\subsubsection{Ruido aditivo en imágenes}

\begin{description}
  \item[Muestra:] lena.pgm
  \item[PSNR inicial:] 27.23 dB
  \item[PSNR final:] 28.62 dB
\end{description}

\begin{figure}[H]
  \centering
  \begin{subfigure}[b]{0.45\textwidth}
    \centering
    \includegraphics[width=\textwidth]{graficos/lena_aditivo_muestra.png}    
    \caption{Antes de atenuar.}
  \end{subfigure}
  ~ 
  \begin{subfigure}[b]{0.45\textwidth}
    \centering
    \includegraphics[width=\textwidth]{graficos/lena_aditivo_atenuar_muestra.png}
    \caption{Después de atenuar.}
  \end{subfigure}
  \caption{Ruido aditivo en imágenes.}
\end{figure}

\begin{figure}[H]
  \centering
  \includegraphics[width=15cm]{graficos/lena_aditivo_atenuar_dct.png} 
  \caption{Muestra en base DCT.}
\end{figure}


\subsubsection{Ruido impulsivo en audio}

\begin{description}
  \item[Muestra:] dopp512.txt
  \item[PSNR inicial:] 15.54 dB
  \item[PSNR final:] 17.55 dB
\end{description}

\begin{figure}[H]
  \centering
  \includegraphics[width=15cm]{graficos/dopp_impulsivo_atenuar_muestra.png} 
  \caption{Muestra en base canónica.}
\end{figure}

\begin{figure}[H]
  \centering
  \includegraphics[width=15cm]{graficos/dopp_impulsivo_atenuar_dct.png} 
  \caption{Muestra en base DCT.}
\end{figure}


\subsubsection{Ruido impulsivo en imágenes}

\begin{description}
  \item[Muestra:] lena.pgm
  \item[PSNR inicial:] 21.53 dB
  \item[PSNR final:] 24.92 dB
\end{description}

\begin{figure}[H]
  \centering
  \begin{subfigure}[b]{0.45\textwidth}
    \centering
    \includegraphics[width=\textwidth]{graficos/lena_impulsivo_muestra.png}    
    \caption{Antes de atenuar.}
  \end{subfigure}
  ~ 
  \begin{subfigure}[b]{0.45\textwidth}
    \centering
    \includegraphics[width=\textwidth]{graficos/lena_impulsivo_atenuar_muestra.png}
    \caption{Después de atenuar.}
  \end{subfigure}
  \caption{Ruido impulsivo en imágenes.}
\end{figure}

\begin{figure}[H]
  \centering
  \includegraphics[width=15cm]{graficos/lena_impulsivo_atenuar_dct.png} 
  \caption{Muestra en base DCT.}
\end{figure}


%%%%%%%%%%%%%%%%%%%%%%%%%%%%%%%%%%%%%%%%%%%%%%%%%%%%%%%%%%%%%%%%%%%%%%%%%%%%%%%
%% Umbralizar                                                                %%
%%%%%%%%%%%%%%%%%%%%%%%%%%%%%%%%%%%%%%%%%%%%%%%%%%%%%%%%%%%%%%%%%%%%%%%%%%%%%%%


\subsection{Umbralizar}

Dado un umbral beta y un índice i, anulamos todos los coeficientes con índice mayor o igual a i y valor absoluto menor o igual a beta.



\subsubsection{Ruido aditivo en audio}

\begin{description}
  \item[Muestra:] ramp1234.txt
  \item[PSNR inicial:] 22.54 dB
  \item[PSNR final:] 24.60 dB
\end{description}

\begin{figure}[H]
  \centering
  \includegraphics[width=15cm]{graficos/ramp_aditivo_umbralizar_muestra.png}
  \caption{Muestra en base canónica.}
\end{figure}

\begin{figure}[H]
  \centering
  \includegraphics[width=15cm]{graficos/ramp_aditivo_umbralizar_dct.png} 
  \caption{Muestra en base DCT.}
\end{figure}


\begin{description}
  \item[Muestra:] dopp512.txt
  \item[PSNR inicial:] 14.31 dB
  \item[PSNR final:] 16.45 dB
\end{description}

\begin{figure}[H]
  \centering
  \includegraphics[width=15cm]{graficos/dopp_aditivo_umbralizar_muestra.png}
  \caption{Muestra en base canónica.}
\end{figure}

\begin{figure}[H]
  \centering
  \includegraphics[width=15cm]{graficos/dopp_aditivo_umbralizar_dct.png} 
  \caption{Muestra en base DCT.}
\end{figure}


\subsubsection{Ruido aditivo en imágenes}

\begin{description}
  \item[Muestra:] lena.pgm
  \item[PSNR inicial:] 27.27 dB
  \item[PSNR final:] 28.08 dB
\end{description}

\begin{figure}[H]
  \centering
  \begin{subfigure}[b]{0.45\textwidth}
    \centering
    \includegraphics[width=\textwidth]{graficos/lena_aditivo_muestra.png}    
    \caption{Antes de umbralizar.}
  \end{subfigure}
  ~ 
  \begin{subfigure}[b]{0.45\textwidth}
    \centering
    \includegraphics[width=\textwidth]{graficos/lena_aditivo_umbralizar_muestra.png}
    \caption{Después de umbralizar.}
  \end{subfigure}
  \caption{Ruido aditivo en imágenes.}
\end{figure}

\begin{figure}[H]
  \centering
  \includegraphics[width=15cm]{graficos/lena_aditivo_umbralizar_dct.png} 
  \caption{Muestra en base DCT.}
\end{figure}


\subsubsection{Ruido impulsivo en audio}

\begin{description}
  \item[Muestra:] dopp512.txt
  \item[PSNR inicial:] 14.87 dB
  \item[PSNR final:] 16.96 dB
\end{description}

\begin{figure}[H]
  \centering
  \includegraphics[width=15cm]{graficos/dopp_impulsivo_umbralizar_muestra.png}
  \caption{Muestra en base canónica.}
\end{figure}

\begin{figure}[H]
  \centering
  \includegraphics[width=15cm]{graficos/dopp_impulsivo_umbralizar_dct.png} 
  \caption{Muestra en base DCT.}
\end{figure}


\subsubsection{Ruido impulsivo en imágenes}

\begin{description}
  \item[Muestra:] lena.pgm
  \item[PSNR inicial:] 21.39 dB
  \item[PSNR final:] 24.92 dB
\end{description}

\begin{figure}[H]
  \centering
  \begin{subfigure}[b]{0.45\textwidth}
    \centering
    \includegraphics[width=\textwidth]{graficos/lena_impulsivo_muestra.png}    
    \caption{Antes de umbralizar.}
  \end{subfigure}
  ~ 
  \begin{subfigure}[b]{0.45\textwidth}
    \centering
    \includegraphics[width=\textwidth]{graficos/lena_impulsivo_umbralizar_muestra.png}
    \caption{Después de umbralizar.}
  \end{subfigure}
  \caption{Ruido impulsivo en imágenes.}
\end{figure}

\begin{figure}[H]
  \centering
  \includegraphics[width=15cm]{graficos/lena_impulsivo_umbralizar_dct.png} 
  \caption{Muestra en base DCT.}
\end{figure}

% \subsection{Generación de ruidos}
% Implementamos el ruido aditivo e impulsivo tanto para imágenes como para sonido.
% \subsubsection{Ruido aditivo}
% La implementación agrega ruido aditivo según la distribución probabilística normal pudiendo variarse su media y varianza para generar diferente cantidad de ruido.

% Este ruido generalmente altera todas las frecuencias de la señal por igual variando su intensidad según los cambios en la varianza.

% \subsubsection{Ruido impulsivo}
% La implementación recorre la señal o la imágen y cambia sus valores por su máximo o mínimo según un valor $p$ asociado a los valores aleatorios obtenidos mediante una distribución probabilística uniforme entre $[0,1]$. Si el valor aleatorio es $< p$ entonces se impulsa la señal al máximo, si es $\geq p $ se impulsa al mínimo, y si cae entre ellos la señal queda intacta.

% \subsection{Eliminación de ruido}
% Implementamos y probamos los métodos de eliminación de ruido \"atenuar\" y \"umbralizar\" descriptos a continuación sobre intervalos, ya que, en las pruebas realizadas, se consiguieron mejoras en las señales en su aplicación a intervalos específicos.

% En el caso de imágenes, el intervalo se tiene en cuenta según su recorrido en forma diagonal, abarcando desde los cosenos con más baja frecuencia hasta los de más alta.

% \subsubsection{Atenuar intervalo}
% Este método de reducción de ruido, multiplica a un intervalo de la señal por una constante k de doble precisión de punto flotante.

% Para el caso de un k menor que uno, la señal se reduce, siendo útil para reducir incrementos de amplitudes en frecuencias.

% \subsubsection{Umbralizar intervalo}
% El método recorre un intervalo la señal, buscando y poniendo en 0 los valores que sean menores al valor absoluto de una constante k que recibe como parámetro.

% La idea de su implementación es utilizarse para poder umbralizar ciertas frecuencias que no suelen pasar de un valor determinado, pudiendo atenuar el ruido de la señal.


%%%%%%%%%%%%%%%%%%%%%%%%%%%%%%%%%%%%%%%%%%%%%%%%%%%%%%%%%%%%%%%%%%%%%%%%%%%%%%%
%% Conclusiones                                                              %%
%%%%%%%%%%%%%%%%%%%%%%%%%%%%%%%%%%%%%%%%%%%%%%%%%%%%%%%%%%%%%%%%%%%%%%%%%%%%%%%


\section{Conclusiones}

Pendiente.


%%%%%%%%%%%%%%%%%%%%%%%%%%%%%%%%%%%%%%%%%%%%%%%%%%%%%%%%%%%%%%%%%%%%%%%%%%%%%%%
%% Apéndice A: Enunciado del Trabajo Práctico                                %%
%%%%%%%%%%%%%%%%%%%%%%%%%%%%%%%%%%%%%%%%%%%%%%%%%%%%%%%%%%%%%%%%%%%%%%%%%%%%%%%

\newpage

\section{Apéndice A: Enunciado del Trabajo Práctico}

\input{enunciado}


\end{document}